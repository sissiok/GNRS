%\documentclass[conference, 11pt]{IEEEtran}
\documentclass[10pt]{article}
%\usepackage{iabproject}
\usepackage{times}
\usepackage{graphicx}
\usepackage{amsmath, amssymb,latexsym}
\usepackage{url}
\usepackage{multicol}
\usepackage[margin=0.5in]{geometry}



\title{Global Name Resolution Service}

\author{Robert Moore\\
\\
Richard P. Martin\\
\and
Feixiong Zhang\\
\\
Yanyong Zhang\\
\and
Kiran Nagaraja\\
\\
Thu D. Nguyen\\
}


\begin{document}
\maketitle

\section{Background}
GNRS provides a distributed, redundant storage network for
GUID$\rightarrow$Network Address (NA) binding values.  By utilizing the latest
secure hashing algorithms and replicating across multiple autonomous systems,
we plan  to scale the system to tens of thousands of nodes while maintaining
very low access times and high network availability.

GNRS is a general-purpose distributed name-value binding system designed to
support low-latency, highly-available binding information for mobile devices
in the next generation Internet.  Specifically, the primary goal is to enable
insertion and retrieval of GUID$\rightarrow$NA binding values across the
global internet in under 100ms.  However, because GNRS is a network
application, it does not need to be bound to any specific network architecture
or implementation, and we have been able to produce a functional prototype
based on the current IPv4 networking standard.

Prior work focused on simulating such a distributed system ~\cite{tamvu-dmap}
across a network of 65,000 autonomous systems (ASes).  Our current effort is
focused on building both a prototype software artifact that implements both
the DMap algorithm as well as supporting realistic workloads.  

%\subsection{Related Work}
%Similar to DNS~\cite{rfc1035} in its functionality, GNRS provides a name (GUID) to address
%(NA) binding service.  However, the two systems differ in significant ways.
%DNS enforces a strict hierarchy of names (e.g., com $\rightarrow$ example.com
%$\rightarrow$ www.example.com), and authoritative servers responsible for
%certain portions of the name system.  In contrast, GNRS utilizes a flat
%address space (or hash space) for GUID values, and any number of servers may
%store the ``authoritative'' bindings (replica servers).
%
%Even more importantly, the current DNS supports binding changes to be
%propagated throughout the system over a period of hours or even days.  While
%sufficient for traditional static networks, this approach cannot support
%highly mobile devices like sensor nodes, mobile phones, or transportation
%vehicles.  To support this level of mobility, where network attachment points
%may change several times per minute, GNRS supports binding changes on the
%order of a few second, even at global distances.
%
%
%As part of our evaluation, we will investigate overall system performance,
%reliability, scalability issues.  In future work, we plan to explore security
%issues, failure recovery scenarios, and optimization possibilities.

\section{Technical Approach}
To achieve realistic network emulation conditions, we have developed both a
software prototype GNRS server as well as testing/emulation framework based on
OMF and utilizing the ORBIT facility at WINLAB~\cite{orbit}.  By combining
these two components, we are able to build and configure a testbed enviroment
in only a few minutes.  This greatly increases our ability to perform
repeatable experiments as well as reducing the effort required to create new
testing scenarios.

%In our work, we present a prototype implementation, written in the
%cross-platform Java programming language, that performs well as an
%experimental tool and as a proof-of-concept reference implementation.  In
%addition to the server itself, we provide multiple test clients, C/C++
%networking headers, and network protocol specifications for independent
%evaluation and implementation.

\subsection{GNRS Reference Implementation}
The new server has been written in Java to provide a hardware- and operating
system-agnostic implementation.  Wherever possible, standard libraries are
utilized to provide the required functionality, and only the application logic
needed to be written by hand.  The server is organized into several individual
modules: network access, GUID mapping, persistent storage, and application
logic.  The application logic serves as a central point of coordination within
the framework of the GNRS server d\text{ae}mon.

The network access component ensures that the GNRS server is able to operate
over any networking layer/technology without changes to the core code.  This
replaceable component currently supports IPv4 and will be written for MF
routing in the near future.

The GUID mapping module, relying partly on the networking implementation,
enables the server to determine the remote GNRS hosts responsible for
maintaining the current bindings of GUID values.  This is similar to the DMap
proposal for using BGP prefix announcements as the basis for GUID hashing.

Persistent storage is handled indepedently from the rest of the server and
exposes only a very simple interface, mapping to the application messages
available in the protocol.  Currently, a BerkeleyDB provides both in-memory
and on-disk storage for GUID bindings.  In the future we will explore external
databases, NoSQL data stores, and other possibilities.

\subsection{Emulation Framework}
In order to reliably test the performance of and identify problems with the
GNRS server, we designed a network emulation framework based on the ORBIT
Management Framework (OMF) 

\section{Summary of Results}
This latest version of the GNRS server provides all of the features described
above (modularity, flexibility, maintainability) and has equivalent
per-instance performance characteristics when compared with the previous C/C++
implementation.  We are currently evaluating the system in small-scale
(200 node) deployments, and plan to scale to thousands of instances in the
future.

\begin{figure}[t]
	\begin{center}
		\begin{tabular}{cc}
			\includegraphics[width=0.5\linewidth]{fig/dmap-query-latency} &
			\includegraphics[width=0.5\linewidth]{fig/clt-lkp-deg} \\
			(a) & (b) \\
		\end{tabular}
		\caption{(a) D-Map simulation results for lookups (queries) using
		different numbers of replicas. (b) GNRS emulation results for lookups
	using K=5 replicas in a network of 200 ASes ranked by degree.}
		\label{fig:results}
	\end{center}
\end{figure}

During our redesign and reimplementation of the server, we have identified
several important issues earlier than if we had not performed such an
extensive review.  This includes the lifetimes of value bindings, caching
opportunities and directives, and how to enable powerful client-server
interactions while hiding the complex details of the system.

\bibliographystyle{unsrt}
\bibliography{paper}

\end{document}
