\vspace{-0.15in}
\section{Related Work}
\label{sec:related}
Given the importance of locator/identifier separation schemes in both current and future networks, various architectures for mapping identifiers to locators have been proposed and studied. Most of the early mapping schemes~\cite{farinacci-alt,jen,jakab,mathy} assumed aggregatable identifier spaces and proposed ideas based on that vantage point. However, this assumption is too restrictive making such schemes not applicable to many recent mainstream proposals such as HIP~\cite{moskowitz}, AIP~\cite{andersen} and MobilityFirst~\cite{mobilityFirst} which propose flat identifiers. Our approach, in contrast, targets a flexible resolution service by not making any assumptions about identifier hierarchy or locator structure.

There are some recent mapping architecture proposals that incorporate flat identifier space such as DHT-MAP~\cite{luo}, SLIMS~\cite{hou}. However these approaches either incur high lookup latency, making it not applicable to highly mobile environment, or high
management overhead which limits scalability. For example, the DHT based scheme in~\cite{luo} can entail up to 8 logical hops introducing an average latency of about 900ms as per their assumptions.

%While providing failure resilience and self-organizing property, DHT-based approaches may lead to excessive access delays due to multi-overlay-hop communication and management overhead.

In contrast, our scheme aims for much lower latencies by employing the one-hop hashing approach and ensures minimum management overhead for feasible deployment on a global scale. We argue that making use of network entities and the IP reachability information already available through the underlying routing infrastructure provides a practical and scalable approach to realize mapping resolvers. Reference~\cite{seattle_Kim} uses a similar in-network hashing scheme to target the different but related problem of name-based routing.

This work also focuses on a global-scale simulation to validate the design, which has been neglected in most of the prior works referenced above. Reference \cite{jakab} is a recent exception which presents a trace based simulation using the iPlane dataset~\cite{madhyastha}. Our simulation approach is more realistic than that of~\cite{jakab} on two counts: (a) We use a larger dataset from DIMES~\cite{shavitt} to extract AS level connectivity and latency information. The DIMES dataset is based on measurements from $\sim$1000 vantage points compared to $\sim$200 for iPlane, resulting in information for about twice the number of ASs as compared to iPlane; (b) To generate resolution lookup events,~\cite{jakab} uses DNS lookup traces from two particular source locations which introduces a significant locality bias in their results. In contrast, we globally distribute lookup source locations by weighting the chances of choosing a particular source location (source AS) in proportion to the available data on number of end nodes near that location. The basic intuition here is to mimic realistic deployment where more lookup requests will be generated from more densely populated areas.

%Existing schemes can be broadly categorized into two groups based on the identifier structure being assumed, namely \emph{aggregatable identifier}~\cite{farinacci-alt,jen,jakab,mathy} and \emph{flat identifier}~\cite{luo,hou}. In sharp contrast with the former group, our scheme is applicable to any any constraint moving away from a centralized control over name and address ownership, which forms a single root of global trust


%The key differences of our designs comparing to existing techniques are independence of identifier and locator structure, single-overlay-hop lookup and



%Seattle make use of

%DNS-based schemes where locator structure are assumed being hierarchical and DHT-based schemes in which a
%incur one (or more than one) of the following limitations: (i) identifier structure dependence, (ii) spaces or hierarchical underlying routing structure,
%One
%Our approach is moving away from a centralized control over name and address ownership, which forms a single root of global trust, to a fully distributed, hence highly scalable, scheme. \arcName~utilizes underlying routing infrastructure to create a network consisting of globally distributed nodes.

%Being a critical component of locator/identifier separation schemes, various architectures for mapping identifier to locator have been proposed and studied since the very introduction of these separation ideas. Most of the early separation schemes~\cite{farinacci-alt,jen,jakab,mathy} implicitly assumed aggregatable identifier spaces and hence the accompanying mapping architecture was designed to make use of this aggregation. However an increasing number of recent mainstream proposals such as HIP~\cite{moskowitz}, AIP~\cite{andersen} and MobilityFirst~\cite{mobilityFirst} argue for flat identifiers by pointing out its fundamental benefits and implementation feasibility. As such, some recent mapping architecture proposals such as DHT-MAP~\cite{luo}, SLIMS~\cite{hou} are designed to work with a flat identifier space, which is also a basic assumption in our scheme.

%    Based upon the trends witnessed by today's Internet, we believe an efficient global naming resolution service should have the following properties.
 %Firstly, it should be highly scalable be able to support billions of mobile devices.
 %Secondly, it should be responsive and provide access latencies as low as tens of milliseconds. Here, accesses include naming entry updates as well as lookups. *** why tens of ms? ***
 %Thirdly, it should be generic and independent of any specific naming structure.
  %Fourthly, it should be backward compatible with current Internet infrastructure.
  %Fifthly, it should be robust, and resilient against random node and network failures.

%Examples of mapping architectures for aggregatable identifiers include LISP-ALT~\cite{farinacci-alt}, which proposes an overlay scheme through which ETRs can broadcast their identifier prefixes globally similar to BGPs functionality, and APT~\cite{jen} which also relies on identifier aggregation to store the entire global mapping table at one or more ``default mappers" inside each AS. More recently, LISP-TREE~\cite{jakab} introduced a DNS based mapping architecture that uses increasing scopes of identifier prefixes to hierarchically arrange specialized mapping servers. In addition to being unfeasible for flat identifier structure, this scheme suffers from the fact that host mobility greatly reduces the effectiveness of any caching scheme, which in turn \emph{is} the main advantage of the DNS. LISP-DHT~\cite{mathy} proposes an opposite approach of arranging the mapping servers -  in a completely distributed manner and uses a Chord like overlay scheme to answer mapping queries. However the basic assumption that each resolver stores a continuous set of identifier mappings and using its highest value as the node identifier makes it particularly unsuited for flat identifiers.

%\cite{ahlgren} describes the key architectural challenges of mapping flat identifiers to locators and points out that while DHT based approaches are promising for local-scope resolution, the inherent problems of large average hop count and substantial overhead in case of churn inhibit its application for global-scope resolution. The DHT based scheme in ~\cite{luo}, for example, can entail up to 8 logical hops introducing an average latency of about 900ms as per the assumptions in~\cite{luo}. Our scheme, in contrast, aims for much lower latencies to ensure feasible deployment on a global scale. \cite{hou} describes an alternative probabilistic caching based scheme for flat identifiers but it requires significant management overhead for collecting flow information and making caching policies.

%A key focus of our project is also on detailed, global-scale validation of our proposal, which has been neglected in most of the prior works referenced above. \cite{jakab} is a recent exception which presents a trace based simulation using the iPlane dataset~\cite{madhyastha} for realistically modeling Internet topology and latencies. Our simulation approach differs from that of~\cite{jakab} on two main fronts: (a) We use the Dimes database~\cite{shavitt} for extracting AS level connectivity and latency graphs which relies on ~1000 vantage points compared to ~200 for iPlane and results in information for about twice the number of ASs as compared to iPlane; (b) To generate lookup requests,~\cite{jakab} uses DNS lookup traces from two particular source locations which might introduce a significant local bias on the resulting data. In contrast, we globally distribute lookup source locations by weighting the chances of choosing a particular source location (source AS) in proportion to the available data on number of end nodes near that location. The basic intuition here is to mimic realistic deployment where more lookup requests will be generated from more densely populated areas.

%Scalability of the current Internet routing protocol has been a core topic of discussion in the networking community in the recent past. While the timeline for expected large scale routing related performance issues have been debated, there is general consensus on the increasing inefficiency and lack of feature support in the existing IP routing protocol~\cite{narten}. Super-linear growth in the number of prefixes being propagated into the DFZ, increase in the rate of routing updates, increase in the demand for multi-homing support and intra-AS traffic engineering are some of the main contributors to this cause. A recent report from Internet Research Task Force - RFC 6115~\cite{li} summarizes the proposed approaches to tackle the scalability problems in the current architecture. Even though the group could not select a single proposal for endorsement, the group reached "rough consensus that separating identity from location is desirable and technically feasible." One of the key points of critique of the multiple proposals on locator-Identifier split architecture, listed in~\cite{li} was the feasibility of a scalable infrastructure for mapping from locator to identifier. In this paper, we explore precisely this topic and present the outline of a novel distributed mechanism for resolving the locator to identifier mappings.
%The Locator/ID Separation Protocol~\cite{farinacci} and other similar proposals [?,?] address the problem associated with using a single namespace, the IP address, to simultaneously express two functions about a device. Instead they propose two different namespaces - an endpoint identifier (EID), which serves as the device identity and a routing locator (RLOC) which denotes its network location.\\

%TBA: detail about LISP\\
%TBA: mention MobilityFirst and how this fits\\
