\section{Introduction}

Since Saltzer's reflections on the distinction of names and addresses in networks~\cite{saltzer}, the idea of separating location and identity of network entities has attracted a plethora of incremental as well as clean-slate proposals for the design of future Internet. The reason for locator/identifier separation being at the core of these proposals is its direct impact on four main requirements of a new Internet architecture as suggested in the Internet Research Task Force (IRTF) design goal document~\cite{li_design}. These key requirements are routing scalability, mobility support, multi-homing support and traffic engineering enhancements, each of which has been shown to be addressable by the splitting of location and identity namespaces. While there is broad agreement on this separation idea~\cite{li}, the implementation proposals widely vary in terms of two main design components - the structures of the locators and identifiers, and the mapping scheme used to resolve the identifiers to locators. For example LISP~\cite{farinacci}, [?] and [?] assume a hierarchical identifier space, HIP~\cite{moskowitz} and MobilityFirst~\cite{mobilityFirst} incorporate a completely flat identifier space and an even different scheme is proposed in MILSA \cite{milsa}, where identifiers are made up of variable flat and hierarchical parts depending on the context. In contrast, most of the mapping scheme proposals such as LISP-ALT~\cite{farinacci-alt}, LISP-DHT~\cite{mathy}, SLIMS [?], and others are implicitly based on the assumption of aggregatable, hierarchical identifier space - they can work with flat identifiers but the announcement overhead and/or the table sizes would become evidently unscalable. 

The second key difference between the different locator/ identifier separation ideas is in their fundamental design goals. The LISP proposal, for instance, emphasizes backward compatibility and thus reuses the 32-bit aggregatable IP addresses as endpoint identifiers. As a result, mobility and multi-homing support do not follow naturally from its design and variations such as LISP-MN have been subsequently introduced for full mobility support~\cite{farinacci-mn}. The MobilityFirst approach, in contrast, is a clean slate future Internet architecture project with a particular focus on supporting large-scale, efficient and robust mobility services. Keeping these features as basic architectural goals impacts the design of the mapping service and introduces stringent requirements on mapping latency and staleness, which we show are not met by existing mapping schemes. We argue that while there's a rich variety of locator/identifier separation ideas covering a number of different design goals, the development of the accompanying mapping schemes has been myopic and predominantly based on the LISP proposal. In this paper, we introduce a new mapping scheme built on the principle of co-operative sharing of the locator/identifier mapping task among all the ASs in the network and show how it satisfies the two design requirements listed above, viz.,(i) Name space agnosticism and (ii) Native support for fast mobility. 

\subsection{The case for a co-operative mapping scheme}

We propose a conceptual shift from ASs `owning' endpoint identifiers and individually storing the mappings of its owned set of identifiers to a network-wide sharing of the identifier-locator mappings independent of the AS boundaries. This means that the identifier to locator mapping for an identifier $X$ in AS $A$ will be stored in a set of foreign ASs $\{B_1, B_2, \ldots, B_k\}$ instead of being stored in $A$ itself. The reason for choosing such a scheme is that we can make $B_is$ as being deterministically derived from $X$ and leverage the inter-domain routing scheme (e.g. BGP) to reach $B_is$ in a single overlay hop. This leads to drastically reduced mapping resolution latency compared to DHT based schemes without requiring any DNS like infrastructure services. The aim of this paper is to evaluate the feasibility of such a scheme, compare its latency and scalability with previously defined mechanisms and also to address the evident concern about incentive - \emph{Why would AT\&T store and manage Verizon's identifiers?} We argue that just like peer-to-peer file sharing systems and TCP congestion control, co-operative schemes that result in common good at small individual cost have a natural incentive mechanism as long as the individual cost of participation is `reasonable'. In particular, the incentives for foul-play, i.e., not storing/answering mapping requests in this case, would depend on the benefit vs. possible penalty of non-compliance. Both technical solutions (such as reputation management in peer-to-peer systems) and non-technical policy bindings (analogous to Network Neutrality arguments) can be invoked to force/persuade ASs to fairly participate in the scheme. However, in this paper, we focus on the performance evaluation and implementation aspects of the scheme leaving the design of the incentive mechanism still open. As we show through simulation based evaluation, our scheme results in `proportional fairness' of mapping assignments, in the sense that as AS which is expected to generate a large number of identifier to locator mapping requests from other ASs, itself will have to store and answer a proportionally large number of mappings. While we address the incentive and privacy issues raised by this ideological shift, our main goal here is to challenge the presumed constraint of fixed, ownership based mechanisms and explore if a more efficient co-operative scheme is feasible. In particular, the key contributions of this paper are: 



%Based on these requirements we introduce a mapping distribution scheme in which, instead of each AS storing and managing its set of identifier-locator tables, each mapping entry is managed by a set of foreign ASs. We explain the reasons behind this choice, point out its benefits and address the evident incentive and privacy issues. Specifically, our approach leverages the locator reachability information (IP reachability in the present Internet context) that is readily available at the network layer, and makes use of an in-network single hop hashing technique to reach the storage point for any given identifier-locator mapping entry with low latency. We make the following key contributions in this paper: 

\begin{enumerate}
\item 
We propose a co-operative identifier to locator mapping scheme that distributes the mapping entries among participating ASs without regard to ownership and AS boundaries. We enumerate its benefits and address the evident problems associated.
%We propose a conceptual shift from ASs `owning' endpoint identifiers and individually storing the mappings of its owned set of identifiers to a network-wide sharing of the identifier-locator mappings irrespective of the AS boundaries. Our scheme results in a natural order of incentives in which an AS which is expected to produce a large number of identifier-locator lookups from other ASs, itself stores and answers a proportionally large number of mappings. We aim to provide reasonings that show the feasibility of the scheme in order to open further research into this line of thought. 

\item 
We introduce a novel single-hop hashing technique that leverages the IP reachability information available at the network layer for low-latency mapping queries. To lookup a mapping in this scheme, one can directly hash the identifier to produce the network address of the AS that stores its mapping. 

\item 
We validate our scheme through a large-scale discrete event simulation based on Internet measurement traces. We show that this scheme achieves low latency values with a 95th percentile value of $\sim$100 ms, a 2x - 5x reduction compared to previously published results. 

\end{enumerate}

The outline of the paper is as follows:


