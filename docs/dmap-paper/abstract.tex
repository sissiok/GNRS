%As more mobile devices are connected to Internet, separating host names from network addresses is becoming a key enabling technique. In this paper we propose a novel architecture to do fast naming resolution service that can provide the mapping from a host name to its network address, thus facilitating this separation. We take a distributed approach by storing the mappings on the routers, and uses consistent hashing to find the routers that store a specific host name. As a result, our naming resolution service is scalable, fast, and reliable. Through detailed simulations with *** routers and *** names, we show that we can achieve *** ms lookup latencies.
This paper presents the design and evaluation of a novel distributed shared hosting approach, DMap, for managing dynamic identifier to locator mappings in the global Internet.  DMap is the foundation for a fast global name resolution service necessary to enable emerging Internet services such as seamless mobility support, content delivery and cloud computing.
%The proposed global name resolution service (GNRS) supports mobility at-scale in the future Internet in which a clean separation of the ``name" of a network object from its ``address" takes place. The GNRS is intended as a fast network-level service which can be queried by both end-points and in-network routers in order to obtain bindings between a GUID (globally unique identifier for the name) and the current routable network address (or addresses).
Our approach distributes identifier to locator mappings among Autonomous Systems (ASs) by directly applying K$>$1 consistent hash functions on the identifier to produce network addresses of the AS gateway routers at which the mapping will be stored. This direct mapping technique leverages the reachability information of the underlying routing mechanism that is already available at the network layer, and achieves low lookup latencies through a single overlay hop without additional maintenance overheads. The proposed DMap technique is described in detail and specific design problems such as address space fragmentation, reducing latency through replication, taking advantage of spatial locality, as well as coping with inconsistent entries are addressed. Evaluation results are presented from a large-scale discrete event simulation of the Internet with $\sim$26,000 ASs using real-world traffic traces from the DIMES repository. The results show that the proposed method evenly balances storage load across the global network while achieving lookup latencies with a mean value of $\sim$50 ms and $95^{th}$ percentile value of $\sim$100 ms, considered adequate for support of dynamic mobility across the global Internet. 