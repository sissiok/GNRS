\section{Concluding Remarks}
\label{sec:conclusion}
In this paper, we presented the concept, design and evaluation of DMap, a scheme for low latency, scalable name resolution service in the future Internet. DMap distributes name to address mappings amongst Internet routers using an in-network single-hop hashing technique that derives the address of the storage router directly from a flat, globally unique identifier. In contrast to other DHT-based techniques, DMap does not require any table maintenance overhead since we use network level reachability information already available through existing routing protocols. In addition, DMap supports arbitrary name and address structures making it more widely applicable than prior techniques. Through a large-scale discrete-event simulation, we show that the proposed DMap method achieves low latencies with a mean value of $\sim$50 ms and 95th percentile value of $\sim$100 ms and good storage distribution among participating routers.

In further work, we plan to consider other variations of the proposed DMap distribution scheme - for example GUIDs can be hashed directly to AS numbers or allocation sizes can be varied to reflect economic incentives at ASs. We also plan to extend the scope of this work by studying a feasible in-network caching methods that builds on top of the basic DMap scheme. Since our scheme interacts with the hosts, the inter-domain routing protocol and the Internet routers, security is a critical requirement at each level. The MobilityFirst project~\cite{mobilityFirst}, takes a holistic approach towards self certification based security, which tie in well into the relevant aspects of our scheme. Our future work plan also includes incorporating the transient effects of BGP updates, misconfigurations and router failures.

On the validation and evaluation front, there is an ongoing effort to implement a proof-of-concept global scale DMap system using the GENI (global environment for network innovation) framework.  A first DMap prototype was demonstrated at the GENI Engineering Conference-12 in Kansas City and efforts are currently under way to fully instrument the latency and overhead measurements necessary to evaluate scalability and performance.  If the GENI experiments successfully confirm DMap performance, there are also further plans to use the proposed technique as part of a complete identifier-based protocol stack in the MobilityFirst future Internet architecture project.


 